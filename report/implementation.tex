%!TEX root = main.tex
\documentclass[main.tex]{subfiles}

\begin{document}
\todo{Introductory paragraph to the chapter.}

\todo{What was considered the ``Implementation Plan'' in the Interim Report will be converted into the implementation itself, describing how the system works, with some diagrams of the call flow, the methods of filtration, and how the entire system is put together, from PSTN interface to UI. This section will go into the details of all the types. Any algorithms will be here, as well as the technical nitty-gritty.

First thing: setup with Obihai and switch and rasbpbx and asterisk and freepbx and router. THen comes the configuration of in/out. Then rerouting to form the "filtration". Then the output side of things. Then the output communicating with the input.
}
\section{Raspberry Pi}
\subsection{OS}
After acquiring the Raspberry Pi 3, the first step was to install an operating system (OS). The research on RasPBX \cite{raspbx} mean that it was the chosen OS. The image was flashed to an SD card for the Raspberry Pi. There is no need for further details of this step.

\subsection{Network Connections}
The next step was to configure the Raspberry Pi 3 to act as a DHCP server \cite{pi-dhcp}, as the online resource suggested. Again, no further details are included as the instructions were followed. The Pi has created its own LAN, and assigns itself the address $192.168.3.1$. This allowed the Obi110 to be connected to the network through the switch.

\section{ObiHai Obi110 and the Phone}
As there is no landline in College to use, the Obi110 was brought home. Combined with a simple standard analogue phone, everything was connected to show that without any additional configuration, the normal phone with the Obi110 works normally.

\section{Asterisk, FreePBX, and the Obi110}
This segment was done using a combination of materials from Bryan Ross \cite{bryanross}, and a little bit from Phil's Blog \cite{obihaiuk}. Ross's tutorial \cite{bryanross} allows RasPBX to interface with the phone line, and the analogue phone. The manipulation of the Obi110 settings was done through its web interface. RasPBX has an Asterisk server running, but has the FreePBX GUI that is accessible through the web as well. Although the steps that were followed were quite similar, they were only followed until the Obi110 and the RasPBX were linked. Ross's material continues to show how he did his time-based system, which is not relevant in this case.
\\\\
To briefly summarise the settings, the first thing is that the Obi110 is able to ``convert'' the landline into an SIP trunk. This trunk is what Asterisk can use to send and receive calls from the outside. This ensures that calls going to the trunk from Asterisk are converted into something the landline can handle, while calls coming in are sent to the trunk.
\\\\
Next, the Obi110's interface with the analogue telephone is configured as an extension. An extension in this context is an endpoint in a VoIP system. In other words, by configuring it as an extension, it can receive calls sent to it from Asterisk, and will convert it into a form that the phone can handle. Phil's Blog contained some important configuration values that were used to ensure a smoother calling process \cite{obihaiuk}. The entire Obi110 configuration is stored as an XML file, which is included in the Appendix \todo{Reference} as it is relevant, but not necessary in the main body.
\\\\
Once this is done, the Asterisk server was configured through the FreePBX GUI. The server was set to interact with the trunk from the Obi110. This would be where all outbound calls are headed. Next, all inbound calls would be routed to the analogue phone extension in the Obi110. This means that calls headed out go through the SIP trunk to the landline, and incoming calls go straight to the phone.
\\\\
At this point, all the connections to the outside are configured and set. While an additional VoIP phone was obtained from the department, the proprietary nature of the phone meant that it was not possible to use it with the open-source software that was running on the Raspberry Pi.

\subsection{Message, Prompt, and Code}
However, according to Figure \ref{fig:callflow}, the incoming call must go to a Welcome message. This is what the Interactive Voice Response (IVR) was used. The incoming calls are routed to an instance of it. This instance contains a message, and reacts according to the input from the caller. By combining the Welcome message with the reverse Turing test, the process is simplified. The message that the caller hears is as follows.

\begin{quote}
	Welcome! This system is monitored using an anti-spam system. Marketing calls and illegal activity will not be tolerated. Please press 1 to continue.
\end{quote}

If the caller knows the weakly secret code (another number, such as 4 or 9), then the call is routed directly to the analogue extension, without any further delays. If the caller fails to press 1, or the code, or does not react, the IVR directs the call to a recorded message as follows, before the call ends.

\begin{quote}
	Unfortunately, there was an incorrect response. Please redial this number and press 1 at the prompt. Thank you!
\end{quote}

If the reverse Turing test is passed (i.e. ``1'' is pressed), then the IVR directs the call to a Call Recording module. While the neither the caller or the user sees this, the Call Recording module, which takes in a call and redirects to an extension, has begun recording the call.

\subsection{Voice Analysis}
\subsubsection{Extraction of Voice}
The reason for the recording of the call is that it is a pre-existing feature that uses the available infrastructure to obtain the audio from the call. The addition of recording devices that need to be plugged-in to the phone worked against the Design Aims previous discussed in the Chapter \ref{chp:deg}.

It was at this point that the need for a switch was noticed. It was obvious that the recording would start the minute the call was put through. However, the recording included the initial ringing, and started before the user had picked up the phone. There was thus no way to determine how long since the user had picked up the phone. Looking at the waveform of the audio in Figure \ref{fig:wave}, it was seen that the ringing had a distinctive pattern, which could be easily identified. With the ringing happening at regular intervals, one method considered was to use a matched filter to the ringing and note its last instance. However, this was not just very involved, but also very computationally intensive.

\subsubsection{Speech Recognition}
\subsection{Metric}
\section{Display}
\subsection{Webpage and Server}
\subsection{Kiosk Mode}

\begin{figure}
	\centering
	\begin{circuitikz} \draw
		 (0, 0) -- (8, 0)
		 (5, 0) to [R, l_=$150\Omega$] (5, 2)
		 (5, 2) to [R, l_=$150\Omega$] (5, 4)
		 (1, 2) node[label={above:To GPIO}] {} -- (5, 2)
		 (5, 6) to [led] (5, 4)
		 (3, 6) to [push button] (3, 2)
		 (0, 6) node[label={above:3.3V}] {} -- (8, 6)
		 (7, 0) node[ground]{} -- (7, 0)
		;
	\end{circuitikz}
	\caption{Modified switch circuit.} \label{fig:initial-circuit}
\end{figure}

\end{document}
