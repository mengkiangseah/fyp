%!TEX root = main.tex
\documentclass[main.tex]{subfiles}

\begin{document}

With the open ended nature of this project, it was essential to define the goals and objectives. These objectives were also the source of the aimThe two primary design aims were targeted at the elderly, while fitting into existing phone setups. The result was the following

The requirements for this project were also the start of the design aims and objectives, which in turn influenced the implementation.

\hl{This section was used to discuss the finer points of what I was aiming to do in the Interim Report, but this will be adapted to discuss the list of requirements for the setup. This ties in nicely with the specification, because flesh out the details of the system. The way this has evolved between the Interim Report and now will also be compared. Currently, this section is approximately 2 pages long. With the additional information, this will realistically be around 3-4 pages as well. }

\section{Intended Audience}
As discussed in the background, the inconvenience of scam and spam calls can affect anyone with a landline. In a survey conducted by Microsoft, almost 70\% of United Kingdom (UK) respondents reported encountering scam calls \cite{microsoft-survey}. Taking into account that this excludes marketing cold calls, the actual proportion of the population impacted by such calls is no doubt higher.
\\\\
However, certain segments of the population are more susceptible than others. The elderly are seen to be more vulnerable, particularly if the scam call deals with technology that they may be unfamiliar with. From the earlier stages of the project, the audience and intended user base was thus established as the elderly. Additionally, knowing that telephone systems vary around the world, the focus was for this project to work in the UK.
\\\\
With the intended audience in mind, this means that the project must be easy to setup and use. However, that criteria is applicable to any consumer group, meaning that this project must be suitable for someone without a rigorous technical background.

\section{Setup}
The project specifically is aimed at landlines because that is the most commonly used form of home telephone. The use of Voice over Internet Protocol (VoIP) phones has not yet gain serious traction in UK. Statistics show that for the 65.1 million people in the UK \cite{ons}, there are still 25.6 million landlines \cite{ofcom}. Additionally, taking into account that this is aimed at the elderly, it makes more sense. Thus, the objective is to have something that will plug into the telephone socket on one end, and the telephone on the other. Combining this with the intended audience, the setup must be able to work with the UK standards of telephony.

\section{Implementation}
The implementation plan was guided by the idea that a piece of software known as Asterisk could be used. Asterisk is like a private branch exchange, or PBX. In simpler terms, it can act as a switch board, by answering, redirecting, and managing calls. This open-source software runs on Linux, and offers a range of options \cite{asterisk}.

\subsection{Connections}
Asterisk accepts and redirects calls, and its input and output format is known Session Initiation Protocol (SIP) \cite{sip}. This means that there needs to be a way to convert the calls from the landline into an SIP stream. Thus, there will need to be a form of landline-SIP adapting connection at the input. The advantage of this is that if and when VoIP becomes more common, the adapter can simply be bypassed.
\\\\
On the output, there are two options. If a normal landline phone is used, then the SIP needs to be converted back into an analogue signal. This can be done with devices known as Analog Telephone Adapters (ATAs) \cite{ata}. However, the alternative would be to use a VoIP phone directly, as the SIP format is the current standard for VoIP phones. It benefits from not needing a conversion step, but will require an additional phone.

\subsection{Filtration}
Between the input from the landline and the output phone will sit the filtration system. A computer will be needed, and given the criteria for a system to ``plug in'', it must also be small. The solution is therefore to use a Raspberry Pi. The Pi will run the Asterisk PBX, which will perform the various filtration steps \cite{raspbx}.
\\\\
Naive filtration methods are simple ones, such as a prompt to press a key before continuing, or the use of blacklists. More involved methods are limited by the fact that for caller input, there are only two options: voice and keytones. For the identification as an unwanted call will require the use voice or speech analysis or more complex code-based challenges.
\\\\
The main difficulty with implementing these methods is not just their robustness or reliability, but how they appear to the caller. If they are too much of a hassle, they will no doubt discourage scammers, but may drive your friends away as well!

\subsection{User Interface}
The filtration of calls is going to be most successful through a combination of methods. Black and whitelists are easy to generate, whether through open-sourced lists maintained by the community, or through simple rules, like no calls from abroad. Known phone numbers from friends, family, or work can be added to ensure that they pass through. This means that there must be some interface which allows numbers to be added and removed. A monitor with a mouse, an app, or a web interface are all possibilities. However, some will require more work, and others may be suitable for younger users, rather than the elderly. This portion of the project will require more consideration.

\section{Deliverables}
There is one main large deliverable from this project, which is a working system that plugs into the landline on one side, and a phone on the other. The system must then prevent unwanted calls to the landline from reach the phone. Its three major components are as follows. While the output element is simply a matter of having a phone, it is included to show the flow of information through the system.
\begin{itemize}
  \item Landline - SIP Conversion
  \item Asterisk-based Filtration System
  \item Output into a phone
\end{itemize}

\end{document}
