%!TEX root = main.tex
\documentclass[main.tex]{subfiles}

\begin{document}

With the open ended nature of this project, it was essential to first define the goals and objectives. These objectives were shaped the way the design of the project took place. This in turn influenced the implementation. As the requirements were determined at the beginning and guided the entire project, the remained relatively unchanged throughout.

% \hl{This section was used to discuss the finer points of what I was aiming to do in the Interim Report, but this will be adapted to discuss the list of requirements for the setup. This ties in nicely with the specification, because flesh out the details of the system. The way this has evolved between the Interim Report and now will also be compared. Currently, this section is approximately 2 pages long. With the additional information, this will realistically be around 3-4 pages as well. }

\section{Intended Audience}
As discussed in Chapter \ref{chp:bac}, the inconvenience of scam and spam calls can affect anyone with a landline. In a survey conducted by Microsoft, almost 70\% of United Kingdom (UK) respondents reported encountering scam calls \cite{microsoft-survey}. However, other non-spam calls are also a nuisance. Taking those into account, such as marketing cold calls, the actual proportion of the population impacted by such calls is certainly higher.
\\\\
However, certain segments of the population are more susceptible than others. The elderly are seen to be more vulnerable, as mentioned previously, particularly if the scam call deals with technology that they may be unfamiliar with. From the earlier stages of the project, the audience and intended user base was thus established as the elderly. Given the geographical location of Imperial College and knowing that telephone systems vary around the world, the focus was for this project to work in the UK with UK landlines.
\\\\
With the intended audience in mind, this means that the project must be easy to setup and use. But upon closer inspection, that objective is actually applicable to any consumer demographic! A complicated and troublesome product is never desirable. In other words, this project must be suitable for someone without a rigorous technical background.

\section{Setup}
The project specifically is aimed at landlines because that is the most commonly used form of home telephone. The use of Voice over Internet Protocol (VoIP) phones has not yet gain serious traction in UK. Statistics show that for the 65.1 million people in the UK \cite{ons}, there are still 25.6 million landlines \cite{ofcom}. Additionally, taking into account that this is aimed at the elderly, it makes more sense. Thus, the objective is to have something that will plug into the conventional telephone socket on one end, and the standard telephone on the other. Combining this with the intended audience, the setup must therefore be able to work with the UK standards of telephony.

\section{Implementation}
The implementation plan was guided by the idea that a piece of software known as Asterisk could be used. Asterisk is like a private branch exchange, or PBX. In simpler terms, it can act as a switch board, by answering, redirecting, and managing calls. This open-source software runs on Linux, and offers a range of options \cite{asterisk}. However, for this to work the aims of the setup, the implementation also has objectives.

\subsection{Connections}
Asterisk accepts and redirects calls, and one of its many input and output formats is the Session Initiation Protocol (SIP) \cite{sip} which was discussed in the Chapter \ref{chp:bac}. This means that there needs to be a way to convert the calls from the landline into an SIP stream. Thus, there will need to be a form of landline-SIP adapting connection at the input.
\\\\
Another point to consider is that this gives the project the advantage of being adaptable to VoIP in the case that the user does indeed use a VOIP system. As VOIP becomes more common, the system can be modifed by simply bypassing the adapter without the need for additional hardware.
\\\\
On the output, there are two options. If a normal landline phone is used, then the SIP needs to be converted back into an analogue signal. This can be done with devices known as Analog Telephone Adapters (ATAs) \cite{ata}. However, the alternative would be to use a VoIP phone directly, as the SIP format is the current standard for VoIP phones. It benefits from not needing a conversion step, but will require an additional phone. Both are viable options.

\subsection{Identification and Filtration}
Between the input from the landline and the output phone will sit the core system responsible for identification and filtration. A computer will be needed, and as the system needs to be ``plugged in'' between the socket and the phone, it must also be relatively compact. As mentioned before, the use of Asterisk will play a strong part in the core system. Even if it is not able to support all the potential identification methods, because it can interface with ATAs and phone lines, it forms a strong and stable base from which to proceed further in the Design and Implementation stages.
\\\\
A vast range of identification and filtration methods are available, all with the objective to prevent unwanted calls from reaching the users. However, all are limited by the fact that when a call is received, there can only be three sources of information: the audio itself, the key tones, and Caller ID (if available). More involved methods going beyond pressing keys will require either audio processing and/or voice analysis.
\\\\
The main difficulty with implementing these methods is not just their robustness or reliability, but how they appear to the caller. If they are too much of a hassle, they will no doubt discourage scammers, but may drive your friends away as well! In short, the core system should be able to identify and filter suspicious calls, but not discourage legitimate callers from staying on the line.

\subsection{User Interface}
The filtration of calls is going to be most successful through a combination of methods. Initially, black and whitelists were considered to be a sizable part of the identification and filtration methods. They are easy to generate, whether through open-sourced lists maintained by the community, or through simple rules, like no calls from abroad. Known phone numbers from friends, family, or work can be added to ensure that they pass through.
\\\\
Additionally, the information received from the caller and the deductions made by the core system need to be communicated to the user. This means that there must be some interface which allows numbers to be added and removed, and for information to be displayed. A monitor with a mouse, an app, or a web interface are all possibilities. However, some may not be suitable for the elderly.
\\\\
With the potential for other methods to be used, the core system ultimately needs some way to communicate with the user.

\section{Deliverables}
Taking all those requirements into consideration, there is one main large deliverable from this project. This is to be a working system that plugs into the landline on one side, and a phone on the other. The system must then identify and prevent unwanted calls to the landline from reaching the phone. While the output element is simply a matter of having a phone, it is included to show the flow of information through the system. Its three major components are shown in Figure \ref{fig:highlevelrequirement}.

% \begin{itemize}
%   \item Landline - SIP Conversion
%   \item Asterisk-based Identification and Filtration System
%   \item Output into a phone
% \end{itemize}

\tikzstyle{block} = [rectangle, minimum width=2cm, minimum height=3cm, text centered, draw=black]
\tikzstyle{input} = [rectangle, minimum width=2cm, minimum height=3cm, text centered, draw=white]
\tikzstyle{output} = [rectangle, minimum width=2cm, minimum height=3cm, text centered, draw=white]

% The block diagram code is probably more verbose than necessary
\begin{figure}[htb]
\centering
	\begin{tikzpicture}[node distance=3cm]
	    % We start by placing the blocks
	    \node [input, align=center] (input) {Input\\from\\Landline};
	    \node [block, align=center, right of=input] (converter) {Landline\\to\\SIP\\Conversion};
	    \node [block, align=center, right of=converter] (core) {Asterisk\\Based\\Core\\System};
	    \node [block, align=center, right of=core] (phone) {SIP\\to\\Landline\\Conversion};
	    \node [output, align=center, right of=phone] (output) {To\\Phone};

	    % Once the nodes are placed, connecting them is easy.
	    \draw [draw,->] (input) -- node[align=center] {} (converter);
	    \draw [->] (converter) -- node {} (core);
	    \draw [->] (core) -- node {} (phone);
	    \draw [->] (phone) -- node[align=center] {} (output);
	\end{tikzpicture}
	\caption{High Level System Requirement Diagram}
	\label{fig:highlevelrequirement}
\end{figure}

\end{document}
