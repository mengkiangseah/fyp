%!TEX root = main.tex
\documentclass[main.tex]{subfiles}

\begin{document}
\pagenumbering{arabic}
\setcounter{page}{1}

While the term ``spam'' first brings to mind junk emails, it can also mean any other kind of unsolicited communications. Unwanted phone calls are no exception, and are a nuisance, as anyone who has received one can tell you. However, they become more than just an inconvenience when they are scam calls. These calls are made by a malicious party with the express intention to defraud, whether in terms of stealing personal information, financial details, or just money.
\\\\
A simple way to prevent these unwanted call from getting through is to filter incoming calls to a phone number. However, more that just a blacklist or whitelist, a more effective solution would be able to determine the legitimacy of each caller. The goal is to create a system with an effective filter to ensure only desirable calls get through.
\\\\
This report outlines what exactly such a system would require, which governs the philosophy behind its design. The implementation discuss the practical aspects of creating the system, and chronicles the challenges encountered and solutions found. The testing and results sections demonstrate the performance of all parts of the implementation, both individually and as a whole. This is then critically evaluated before the conclusion of the report.  
\\\\
\hl{Content to be added: More about the problem and solution. Some additional context, particular a source to back up the impact and extent of fraudulent calls, both economically and maybe emotionally? Report structure to follow as well. This will lead up to the Introduction not exceeding a single page.}

\end{document}
