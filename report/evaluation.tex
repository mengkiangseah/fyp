%!TEX root = main.tex
\documentclass[main.tex]{subfiles}

\begin{document}
There are a number of areas to evaluate in this project. First, there is the examination of the success of the deliverables. Next, from the wide choice in identification and filtration methods, is the critique of the methods used. The type of testing performed and the analysis of the results offer more insight into the success of the identification and filtration methods. All these are considered in looking possible improvements to this project, and finally, there is a discussion of any other factors to keep in mind.
\\\\
The purpose of this section is to reflect on how well the project's objectives were achieved. As part of that, the ``aims'' in this project were tested as mentioned in Section \ref{sec:goals} and analysed in Section \ref{sec:basic-eval}, but there is also the consideration of the Design Aims that were raised at the start of Chap \ref{chp:des}. For reference, they are repeated here again.

\begin{enumerate}
	\item Fits into existing setups
	\item Self contained
	\item Functioning without needing input
	\item Not too difficult for a genuine caller to get through
\end{enumerate}

\section{Achievement of Basic Functionality}\label{sec:basic-eval}

Of all the mini-goals attempted, all were successful bar one: the emergency bypass.


\section{Testing}
The testing of the various methods was based around the three criteria used in the work of Tu et al. \cite{cisco}. This was split into the

\section{Effectiveness of Methods used and Not used}


\section{Improvements}

\section{Other Considerations}

\subsection{Cost}
The overall cost of this system cannot be exorbitant. The trouble with this estimation is that the raw cost of materials in this project is not the average cost, were such a system to be commercialised. There are economies of scale, partnerships, and the cost of development in the prototype as opposed to the cost of labour in a marketable version. This is in contrast to fixed raw material costs, for example, which are not a concern here.
\\\\
Language
No emergency
Easy?
Did it work?
Improvements
Voice Metric
Not fully isolated
Randomise
Varying scams in vogue

Their methods: robustness test all theoretical, usability, deployabilty

the declining rates of landline usage in the UK, combined with the increased technological literacy in younger generations, the implementation’s direct usage may be less attractive in the future. However, the filtration methods and call evaluation flow may prove to be scalable and transferable to other cases, such as

Alternatives?

Design Aims

Cost I guess
Testing methods - lack of huge data

Declining rates of landline usage?

vs. Simply speaking to old people? Outreach? A non-STEM apraoch to the prblem????


Testing methods themselves evaluated

\hl{This section assesses how well the objectives were achieved. This means looking at what was achieved, and comparing that to the requirements captured in the previous chapter. I will discuss why certain things were achieved, or why failures happened. I will also discuss how improvements could have been made with certain elements, or if the design choices were ultimately successful and why/why not. I expect this to take between 3-4 pages.}

\end{document}
