\documentclass[12pt,a4paper]{report}
\usepackage[utf8]{inputenc}
\usepackage[english]{babel}
\usepackage{mathtools}
\usepackage{amsfonts}
\usepackage{upgreek}
\usepackage{amssymb}
\usepackage{graphicx}
\usepackage[font=footnotesize,labelfont=bf]{caption}
\usepackage[font=footnotesize,labelfont=bf]{subcaption}
\usepackage{lmodern}
\usepackage[left=2cm,right=2cm,top=2.5cm,bottom=2.5cm]{geometry}
\usepackage{fancyhdr}
\usepackage{eurosym}
\usepackage{dcolumn}% Align table columns on decimal point
\usepackage{bm}% bold math
\usepackage{booktabs}
\usepackage{multirow}
\usepackage{framed}
\usepackage{ulem}
\usepackage{mystyle}
\usepackage{lipsum}

\usepackage{xcolor}
\usepackage{soul}

\usepackage{wrapfig}
\usepackage{epstopdf}
\usepackage{url}
\usepackage{etoolbox}
\patchcmd{\thebibliography}{\section*{\refname}}{}{}{}
\makeatletter
% \patchcmd{\chapter}{\if@openright\cleardoublepage\else\clearpage\fi}{\par}{}{}
\makeatother

%Stuff I have added

%Reformat some spacing and sizing around titles
\usepackage{titlesec}
\titleformat{\chapter}[display]
  {\normalfont\Large\bfseries}
  {\chaptertitlename\ \thechapter}
  {20pt}
  {\LARGE}
\titlespacing{\chapter}{0pt}{0pt}{12pt}

%New command to get rid of the "Chapter X" at the beginning of every chapter while maintaining
%a chapter count for the table of contents
%1st input is the counter for the chapter, second is the chapter name
\newcommand{\mychapter}[2]
{
    \setcounter{chapter}{#1}
    \setcounter{section}{0}
    \chapter*{#2}
    \addcontentsline{toc}{chapter}{#2}
}

\usepackage{float}

\usepackage{enumitem}

%Get rid of auto indent for paragraphs
\newlength\tindent
\setlength{\tindent}{\parindent}
\setlength{\parindent}{0pt}
\renewcommand{\indent}{\hspace*{\tindent}}

%End of stuff I have added

\setcounter{tocdepth}{2}
\setcounter{secnumdepth}{3}
\begin{document}
\begin{titlepage}
                % \newgeometry{top=25mm,bottom=25mm,left=38mm,right=32mm}
                \setlength{\parindent}{0pt}
                \setlength{\parskip}{0pt}
                % \fontfamily{phv}\selectfont

                {
                                \Large
                                \raggedright
                                Imperial College London\\[17pt]
                                Department of Electrical and Electronic Engineering\\[17pt]
                                Final Year Project Report\\[17pt]

                }

                \rule{\columnwidth}{3pt}
                \vfill
                \centering
                  \includegraphics[width=0.7\columnwidth,height=60mm,keepaspectratio]{icl.jpg}
                \vfill
                \setlength{\tabcolsep}{0pt}

                \begin{tabular}{p{40mm}p{\dimexpr\columnwidth-40mm}}
                                Project Title: & \textbf{An Intelligent Telephone Interceptor for UK Landlines
} \\[12pt]
                                Student: & \textbf{Meng Kiang Seah} \\[12pt]
                                CID: & \textbf{00699092} \\[12pt]
                                Course: & \textbf{EEE} \\[12pt]
                                Project Supervisor: & \textbf{Dr. Thomas J. W. Clarke} \\[12pt]
                                Second Marker: & \textbf{Dr. Jesus Rodriguez Manzano} \\
                \end{tabular}
\end{titlepage}

% For final report

\begin{acknowledgments}
  Acknowledgements will go here. Supervisor, lab people, and voice artist.
\end{acknowledgments}

\begin{abstract}
This project was focused on creating a system that would intercept landline calls in the UK with the goal of eliminating, or reducing the number of unwanted spam, nuisance, or scam calls. The impact of these calls is greater in certain demographics who are more susceptible, notably the elderly.
\\\\
The system was designed to allow easy connection to an existing landline phone setup. To allow filtering and analysis with a computer, conversion from the analogue signals to a digital stream, and vice-versa was done by integrating commercially available products with free open-source software.
\\\\
A filtration method and flow were designed to remove cold calls, while running voice analysis on unknown callers to determine the likelihood of an illegitimate call. It sets up a challenge upon receiving a call to remove robot calls, separating unknown callers into three categories with voice analysis, and requesting irresponsive callers to redial. The system status was relayed through a simple colour-coded user interface.
\\\\
Overall, the setup showed its feasibility and functionality, while taking into account the limitations of an audience less familiar with technology for interfacing and setup.\end{abstract}

\pagenumbering{roman}
\tableofcontents
\newpage

\mychapter{1}{1. Introduction}
\pagenumbering{arabic}
\setcounter{page}{1}
While the term ``spam'' first brings to mind junk emails, it can also mean any kind of unsolicited communications. Unwanted phone calls are no exception, and are a nuisance, as anyone who has received one can tell you. However, they become more than just an inconvenience when they are scam calls. These calls are made by a malicious party with the express intention to defraud, whether in terms of stealing personal information, financial details, or just money.
\\\\
A simple way to prevent these unwanted call from getting through is to filter incoming calls to a phone number. However, more that just a blacklist or whitelist, a more effective solution would be able to determine the legitimacy of each caller. The goal is to create a system with an effective filter to ensure only desirable calls get through.
\\\\
\hl{Content to be added: More about the problem and solution. Some additional context, particular a source to back up the impact and extent of fraudulent calls, both economically and maybe emotionally? Report structure to follow as well. This will lead up to the Introduction not exceeding a single page.}

\mychapter{2}{2. Background Research}
This background research section is an initial outline. A lot of the research done on this project stemmed from the implementation phases as a requirement for certain configuration issues. In terms of academic level sources, these were difficult to find given the consumer-oriented nature of the system. Additionally, with the use of the Obi110, the need for a in-depth understanding has not been needed.

\todo{Testing.}

\section{Telephone Systems}
\subsection{Analogue}
Telephones are deceptively simple systems. Examination of a landline plug in the UK will reveal that only 2 of the many pins are actually connected! The way this system works it that both the power and signal are carried on the same wires. In the UK, the voltage between the two is at 50V DC, although this can vary \cite{telephone}.
\\\\
For a telephone in a house to work, it must interact with the exchange, which is run by the telephone service provide. When the phone is picked up, the voltage across the line changes. Older systems used this as the basis of the pulse dialing method, but the current modern way is to use DTMF, or Dual Tone multifrequency. Instead of relying on voltage changes, each key when pressed generates a two-frequency tone that the exchange is able to interpret \cite{telephone}.
\\\\
When the exchange has a call to send to a telephone, it transmits a AC (alternating current) signal that is interpreted as the ringing sound \cite{telephone}.

\section{Existing Filtration Methods}
\subsection{Consumer Level}
There are a number of existing methods available. The Telephone Preference Service (TPS) \cite{tps} is a opt-out service that registers official desires not to be contact, consumer rights groups, such as Which? \cite{which} also offer such services. Most telephone service provides also offer call filtration services, usually for an additional fee.

\subsection{Academic Research}
There are a number of papers that have been published regarding the spam calls, such as the work of Chaisamran et al. \cite{chaisa}, Wu et al. \cite{wu} and Heo et al. \cite{heo}. However, a lot of them look at the management side of things. Their filtration systems analyse data at an exchange level to determine fraudulent numbers. This method is not suitable for this application as the system must function independently, with only the call itself as a source of information. Looking through IEEE Xplore unfortunately has not yieled as much information as desired.

\mychapter{2}{2. Project Specification}
% \section{Primary Goals}
\section{Audience}
The inconvenience of scam and spam calls can affect anyone with a landline. In a survey conducted by Microsoft, almost 70\% of United Kingdom (UK) respondents reported encountering scam calls \cite{microsoft-survey}. Taking into account that this exclude marketing cold calls, the actual proportion of the population impacted by such calls is no doubt higher.
\\\\
However, certain segments of the population are more susceptible than others. The elderly are seen to be more vulnerable, particularly if the scam call deals with technology that they may be unfamiliar with. The primary audience of this project is therefore the elderly, which means that it must be easy to setup and use. That criteria however, is applicable to any consumer group, means that this project must be suitable for someone without a rigorous technical background.

\section{Setup}
The project specifically is aimed at landlines because that is the most commonly used form of home telephone. The use of Voice over Internet Protocol (VoIP) phones has not yet gain serious traction in UK. Statistics show that for the 65.1 million people in the UK \cite{ons}, there are still 25.6 million landlines \cite{ofcom}. Additionally, taking into account that this is aimed at the elderly, it makes more sense. Thus, the objective is to have something that will plug into the telephone socket on one end, and the telephone on the other.

\section{Implementation}
The implementation plan was guided by the idea that a piece of software known as Asterisk could be used. Asterisk is like a private branch exchange, or PBX. In simpler terms, it can act as a switch board, by answering, redirecting, and managing calls. This open-source software runs on Linux, and offers a range of options \cite{asterisk}.

\subsection{Connections}
Asterisk accepts and redirects calls, and its input and output format is known Session Initiation Protocol (SIP) \cite{sip}. This means that there needs to be a way to convert the calls from the landline into an SIP stream. Thus, there will need to be a form of landline-SIP adapting connection at the input. The advantage of this is that if and when VoIP becomes more common, the adapter can simply be bypassed.
\\\\
On the output, there are two options. If a normal landline phone is used, then the SIP needs to be converted back into an analogue signal. This can be done with devices known as Analog Telephone Adapters (ATAs) \cite{ata}. However, the alternative would be to use a VoIP phone directly, as the SIP format is the current standard for VoIP phones. It benefits from not needing a conversion step, but will require an additional phone.

\subsection{Filtration}
Between the input from the landline and the output phone will sit the filtration system. A computer will be needed, and given the criteria for a system to ``plug in'', it must also be small. The solution is therefore to use a Raspberry Pi. The Pi will run the Asterisk PBX, which will perform the various filtration steps \cite{raspbx}.
\\\\
Naive filtration methods are simple ones, such as a prompt to press a key before continuing, or the use of blacklists. More involved methods are limited by the fact that for caller input, there are only two options: voice and keytones. For the identification as an unwanted call will require the use voice or speech analysis or more complex code-based challenges.
\\\\
The main difficulty with implementing these methods is not just their robustness or reliability, but how they appear to the caller. If they are too much of a hassle, they will no doubt discourage scammers, but may drive your friends away as well!

\subsection{User Interface}
The filtration of calls is going to be most successful through a combination of methods. Black and whitelists are easy to generate, whether through open-sourced lists maintained by the community, or through simple rules, like no calls from abroad. Known phone numbers from friends, family, or work can be added to ensure that they pass through. This means that there must be some interface which allows numbers to be added and removed. A monitor with a mouse, an app, or a web interface are all possibilities. However, some will require more work, and others may be suitable for younger users, rather than the elderly. This portion of the project will require more consideration.

\section{Deliverables}
There is one main large deliverable from this project, which is a working system that plugs into the landline on one side, and a phone on the other. The system must then prevent unwanted calls to the landline from reach the phone. Its three major components are as follows. While the output element is simply a matter of having a phone, it is included to show the flow of information through the system.
\begin{itemize}
  \item Landline - SIP Conversion
  \item Asterisk-based Filtration System
  \item Output into a phone
\end{itemize}



\newpage
\mychapter{4}{4. Implementation Plan}
\section{Current Progress}
\subsection{Landline-SIP Input}
Originally, the idea of interfacing directly with the telephone line was considered. A series of relays, combined with Analogue to Digital Converters (ADCs) and Digital to Analogue (DAC) could conceivably interact with a normal landline phone to feed the information into the Raspberry Pi. The difficulty with this method is that to use the full range of Asterisk functionality, the inputs have to be in SIP format. The conversion process itself would be very time-consuming.
\\\\
A workaround considered was the Cisco SPA3102 \cite{spa3102-specs} as an adapter. However, this idea was quickly scrapped for a number of reasons. First, the device was no longer supported by Cisco as it had reached its end-of-sale dates. This also meant that prices were extremely high, in the order of hundreds of dollars \cite{spa3102-amazon}.
\\\\
Fortunately, an alternative was found. This was the Obi110, from Obihai \cite{obi110-specs}. This device also had the ability to interface with a PBX and a landline simultaneously. In fact, some users on blogs have written how they have combined the Obi110 with the Raspberry Pi running Asterisk \cite{freepbx} \cite{bryanross}. Discussion of the work of these hobbyists was considered material for the implementation section rather than the background research as none of them applied any non-naive call filtration methods.
\\\\
Approval for the purchase device was received from the Finance department and the device has since been obtained. However, it has yet to be configured.

\subsection{Asterisk Based Filtering}
The Asterisk for Raspberry Pi website, as cited previously, offers images for a complete Raspbian-based operating system (OS) with Asterisk configured on it. This was used and installed on a Raspberry Pi 3. The configuration experimentation on this platform is currently ongoing.

\subsection{Output}
Instead of buying a VoIP phone, spare one was obtained from the department, which solves this segment quite neatly. However, it will require some configuration for it to work.

\section{Future Progress Plan}
\subsection{Work Needed}
The basic implementation, as a barebones setup, is a simple passthrough of calls from the landline through to the adapter, to the Raspberry Pi, and finally to the output phone. The remaining work needed for the basic requires the Obi110 to be configured, and the Asterisk on the Pi to be configured as well.
\\\\
The filtration systems in mind include a series of simple naive ones, such as a keypad button challenge, a blacklist and a firm warning for the caller that the system is monitored. Only if these are successfully implemented will more complex methods be used, such as a voice analysis.
\\\\
It is at this point that more qualitative methods can be explored. This includes analysing how quickly the caller answers the phone, or how fast they speak. These can be linked to the trustworthiness, and the system can return a confidence rating for the legitimacy of the call.

\subsection{Additional Features}
\subsubsection{Emergency Calls}
A common issue with VoIP systems is their ability to process emergency calls. And as mentioned above, the phone line is a powered one that is independent of the electricity in a flat. In a blackout, if only the power fails, phone calls are still possible. With so many devices between the phone and the socket, not only can it not be powered from the phone line, but if any one device fails, the entire system goes down. The solution is to have a relay or a bypass that can be used in an emergency, a power failure, or a device malfunction.

\subsubsection{User Interface}
The method for the user interface will need to be determined. The current OS runs without an graphical user interface (GUI), meaning that an additional interface is necessary. This will need to be created, and how it will be presented (web interface or through the use of a screen) will require more experimentation.

\subsubsection{Answering Machine and Caller ID}
These two features are commonly found on conventional home phones. There should be an attempt to include them in the final system as the goal is to add the filtration to a landline, and not to compromise any existing functionality.

\section{Timeline}
In planning the timeline, a number of dates were kept in mind. There are two exams in May, on the 4th and the 8th. Thus the additional of exam studying time. The draft report is due on the 5th of June and the final report on the 21st of June. Thus, the project is broken down into segments and the estimate completion date, along with a rough outline of the time needed, as Table \ref{tbl:plan} shows.

\begin{table}[H]
\centering
\resizebox{\linewidth}{!}{%
\begin{tabular}{|l|l|l|l|}
\hline
\textbf{Task}                             & \textbf{Start Date} & \textbf{Completion Date}  & \textbf{Time Required} \\ \hline
Inputs and outputs configured             & Immediately         & 12th Feb                  & 2 weeks part-time \\
Passthrough system working                & 12th Feb            & 26th Feb                  & 2 weeks part-time \\
Naive methods implemented                 & 26th Feb            & 31st Mar                  & 5 weeks part-time \\
Complex methods                           & 31st Mar            & 16th Apr                  & 2 weeks full-time \\
Exam studying                             & 16th Apr            & 8th May                   & 3 weeks full-time \\
Complex methods and report writing        & 8th May             & 24th May                  & 2.5 weeks full-time \\
Report writing                            & 24th May            & 5th June                  & 1.5 weeks full-time \\
Additional features and report editing    & 5th June            & 21st June                 & 2.5 weeks full-time \\
Demonstration and final additions         & 21st June           & Presentation Day          & 1 week full-time \\ \hline
\end{tabular}}
\caption{Project Timeline Planning}
\label{tbl:plan}
\end{table}

\section{Fallbacks and Risks}
Certain tasks in Table \ref{tbl:plan} are more guaranteed to succeed than others. For example, getting the passthrough system working is more likely that not to succeed, given the range of hobbyists who have already succeeded and posted proof of their successes. The uncertainty lies in the filtration methods. In fact, all the tasks are arranged in incremental order.
\\\\
The naive methods of passcode challenging, warnings, and lists are first attempted because they are easier to implement and help to determine if complex methods are possible. At the very worst, even if all further attempts fail, the very basic methods will remove robotic calls. However, the reason why it is allocated so much time is because it is the first attempt at configuring the Asterisk software beyond simple call routing, and means there is an additional level of complexity. This, combined with how crucial it is as a first step to filtration means it is given a high level of priority and time.
\\\\
The exact implementations of the complex methods are given quite a long while because it is that segment which will set this system apart from existing solutions available on the market. This is also why it is given as much full-time work as is possible. The time allocated to this section goes on all the way until right before the final report is due. The ``exam break'' is taken into account because, as the project guide says \cite{guide}, neglecting exams for the final year project is a bad idea.
\\\\

\mychapter{5}{5. Evaluation Plan}
\section{Criteria}
The system can be evaluated in a number of ways.
\subsection{Basic Functionality}
The phone system must function, at the bare minimum, to remove robot cold calls. This is the simplest kind of call to avoid, with a simple challenge that automatic calls are not programmed to handle, which is why it is the baseline. Additionally, this functionality must be added to that of a normal phone. This means all other things a normal landline can do must still be possible, such as making outgoing calls or dialing emergency numbers. Success in this criterion is roughly a yes/no situation
\subsection{Cost}
The overall cost of this system cannot be exorbitant. The trouble with this estimation is that the raw cost of materials in this project is not the average cost, were such a system to be commercialised. There are economies of scale, partnerships, and the cost of development in the prototype as opposed to the cost of labour in a marketable version. This qualitative criterion means that the overall cost of project must one that can be lowered through mass-production. This is in contrast to fixed raw material costs, for example, which are not a concern here.
\subsection{Robustness}
The bottom line for this is, "Does it work well?" Can the system prevent unwanted calls from getting through? There are ways to measure this. From previous modules in Machine Learning, one way is to not just at the rate at which the system is successful at filtering. There are two other kinds, the false positive and the false negative. In this situation, a false positive is a legitimate call that is erroneously flagged as a scam, while a false negative is a spam call marked as legitimate. When it comes to calls, a scam call has a small chance of being harmful, while a missed legitimate call is guaranteed to be troublesome. Hence, when looking at this, the system must understand that there are two ``levels'' of wrong.
\subsection{Ease of Caller}
A highly robust way to prevent unwanted calls is for every legitimate caller to enter a pin code before continuing. Naturally, while this would prevent scammers, it does make a call quite a hassle. The ease of the caller is in contrast to the robustness of the system. It must still be easy enough for say a friend to call in, without feeling overly scrutinised by a computer.
\subsection{User Interface}
There must be a way for the user to interact with the system to input their own lists or blocked numbers. They must also be able to configure certain options, such as no calls from aboard, or at certain times of the day. There are a multitude of options, and all must be easily configurable to someone without technical knowledge or training. At the very most, a simple guide booklet could be required, but would have to be produced as part of this.
\subsection{Alternatives}
There are other systems currently on the market that aim to reduce the number of unwanted calls. This system must be able to compete with these other alternatives in terms of cost, functionality, effectiveness and ease of use. Again, this criterion is very qualitative and can depend on what other services/devices are put onto the market between now and June.
\section{Demonstration}
The final demonstration for this project will not be able to involve the landline, as there are no landlines in the college. The current solution is to demonstrate the Landline - SIP Conversion independently, before using a VoIP input directly into Asterisk for the actual demo.


the declining rates of landline usage in the UK, combined with the increased technological literacy in younger generations, the implementation’s direct usage may be less attractive in the future. However, the filtration methods and call evaluation flow may prove to be scalable and transferable to other cases, such as


% No need to touch this.
\newpage
\mychapter{7}{7. References}
\begingroup
   \def\chapter*#1{}
\bibliographystyle{plain}
\bibliography{fyp_refs}
\endgroup

\end{document}
